\documentclass[a4paper,twocolumn]{article}

\usepackage[sc]{mathpazo} % Use the Palatino font
\usepackage[T1]{fontenc} % Use 8-bit encoding that has 256 glyphs
\usepackage[utf8]{inputenc} % Use utf-8 as encoding
\linespread{1.05} % Line spacing - Palatino needs more space between lines
\usepackage{microtype} % Slightly tweak font spacing for aesthetics

\usepackage[spanish]{babel} % Language hyphenation and typographical rules
%\usepackage[galician]{babel} % Change to this if using galician

\usepackage[hmarginratio=1:1,top=32mm,columnsep=20pt]{geometry} % Document margins
\usepackage[hang, small,labelfont=bf,up,textfont=it,up]{caption} % Custom captions under/above floats in tables or figures
\usepackage{booktabs} % Horizontal rules in tables

\usepackage{enumitem} % Customized lists
\setlist[itemize]{noitemsep} % Make itemize lists more compact

\usepackage{abstract} % Allows abstract customization
\renewcommand{\abstractnamefont}{\normalfont\bfseries} % Set the "Abstract" text to bold
\renewcommand{\abstracttextfont}{\normalfont\small\itshape} % Set the abstract itself to small italic text

\usepackage{titlesec} % Allows customization of titles
\renewcommand\thesection{\Roman{section}} % Roman numerals for the sections
\renewcommand\thesubsection{\Alph{subsection}} % roman numerals for subsections
\titleformat{\section}[block]{\large\scshape\centering}{\thesection.}{1em}{} % Change the look of the section titles
\titleformat{\subsection}[block]{\large}{\thesubsection.}{1em}{} % Change the look of the section titles

\usepackage{fancyhdr} % Headers and footers
\pagestyle{fancy} % All pages have headers and footers
\fancyhead{} % Blank out the default header
\fancyfoot{} % Blank out the default footer
%\fancyhead[C]{Running title $\bullet$ May 2016 $\bullet$ Vol. XXI, No. 1} % Custom header text
\fancyfoot[C]{\thepage} % Custom footer text

\usepackage{titling} % Customizing the title section

\usepackage{hyperref} % For hyperlinks in the PDF

\usepackage{float}

% PAQUETES PROPIOS QUE NO SON DE LA PLANTILLAS
\usepackage{listings} % Para insertar codigo
\usepackage{multicol}
\usepackage{color}

\definecolor{dkgreen}{rgb}{0,0.6,0}
\definecolor{gray}{rgb}{0.5,0.5,0.5}
\definecolor{mauve}{rgb}{0.58,0,0.82}
\lstset{
  language=C,
  aboveskip=3mm,
  belowskip=3mm,
  showstringspaces=false,
  columns=flexible,
  basicstyle={\small\ttfamily},
  numbers=none,
  numberstyle=\tiny\color{gray},
  keywordstyle=\color{blue},
  commentstyle=\color{dkgreen},
  stringstyle=\color{mauve},
  breaklines=true,
  breakatwhitespace=true,
  tabsize=3
}
\usepackage{graphicx}
\usepackage{hyperref}
\usepackage[backend=bibtex]{biblatex}
\bibliography{database.bib}

%----------------------------------------------------------------------------------------
%	TITLE SECTION
%----------------------------------------------------------------------------------------

\setlength{\droptitle}{-4\baselineskip} % Move the title up

\pretitle{\begin{center}\huge\bfseries} % Article title formatting
	\posttitle{\end{center}} % Article title closing formatting

\title{Informe Practica 1} % Article title
\author{%
	\textsc{Pablo Landrove Pérez-Gorgoroso, Pablo Díaz Viñambres} \\[1ex] % Your name
	\normalsize Arquitectura de Computadores\\
	% \normalsize Grupo XX \\ % Your institution
	\normalsize \{pablo.landrove, pablo.diaz.vinambres\}@rai.usc.es % Your email address
	%\and % Uncomment if 2 authors are required, duplicate these 4 lines if more
	%\textsc{Jane Smith}\thanks{Corresponding author} \\[1ex] % Second author's name
	%\normalsize University of Utah \\ % Second author's institution
	%\normalsize \href{mailto:jane@smith.com}{jane@smith.com} % Second author's email address
}
\date{\today} % Leave empty to omit a date
\renewcommand{\maketitlehookd}{%
	\begin{abstract}
		\noindent En este informe resumiremos los resultados obtenidos durante la práctica 1 de Arquitectura de Computadores. En general, daremos una comparación de rendimiento de un programa de reducción en el que se han alojado los datos en diferentes posiciones de memoria. Demostraremos así como una buena localidad en el uso de la caché puede aumentar el rendimiento de un programa \\\mbox{}\\
		 \textbf{\textit{Palabras clave}: Memoria caché, localidad espacial, rendimiento de programas\ldots}
	\end{abstract}
}

%----------------------------------------------------------------------------------------

\begin{document}
	
	% Print the title
	\maketitle
	
	%----------------------------------------------------------------------------------------
	%	ARTICLE CONTENTS
	%----------------------------------------------------------------------------------------
	
	\section{Introducción}
	
	En esta práctica el objetivo es mejorar el rendimiento de un programa sencillo escrito en C optimizando el acceso y la forma de los accesos a la caché del procesador.	Cada procesador cuenta con varios niveles de memoria caché de acceso rápido pero baja capacidad, lo cuál permite que el número medio de ciclos de reloj para cada acceso a memoria sea mucho menor que si tuviéramos que ir a la memoria principal en cada consulta. Reducir el número de lineas usadas y optimizar el orden de acceso a ellas nos dára un aumento de rendimiento notable en cualquier programa. \\
	
	Demostraremos que al reducir la localidad de los datos consultados, aumenta el número medio de ciclos por acceso a dato. Al variar los parámetros L y D explicados a continuación cambiaremos el número de líneas consultadas y el espacio de memoria entre cada dato, aumentando estos valores. \\
	
	En las siguientes secciones, explicaremos más detallamente en que consiste el programa y la metodología de las pruebas, daremos las características del ordenador utilizado y presentaremos los resultados en forma de graficas comentadas. Por ultimo, deduciremos que factores afectaron consistentemente al rendimiento del programa. \\
	
	%------------------------------------------------
	\section{Descripción del problema}
	% Aqui meter codigo del problema, explicacion de como funciona
	El código a optimizar consiste en un programa en C, compilado mediante \texttt{gcc} y con optimizaciones en algunas de las pruebas (en el guión se dice que no se usen, y los resultados principales no lo hacen, pero incluimos también una pequeña comparativa para ver cuánto se reduce el número de ciclos medios cuando se usan). \\
	Este programa realiza una operacion de reducción de suma en el array \texttt{A}, de \texttt{R} elementos de tipo \texttt{double} separados entre ellos por \texttt{D} variables double que no se acceden. Esta operación se realiza 10 veces y después se calcula el valor medio de las 10 mediciones. El programa principal en el que esto fue implementado es \textbf{main.c}, insertado an el apéndice A con comentarios que ayudan a entender que hace cada parte del código. \\
	\newpage
	\section{Metodología de las pruebas}
	\subsection{Parámetros de las pruebas}
	% Aqui meter como se hacen las pruebas (i, D, L, R...)
	Para evitar realizar cada prueba manualmente introduciendo distintos argumentos por consola, hemos generado un script de instrucciones \texttt{bash} (\texttt{run\_tests.sh}) para realizar las pruebas y guardarlas en archivos de texto con un solo comando. Ademas en este script llamamos a otros programas como \texttt{mediana.c}, cuyo uso se explica más adelante. \\
	
    En este script hay 3 argumentos de entrada principales: \\
    \begin{itemize}
        \item \texttt{D\_VALUES}: Los 5 valores de D entre 1 y 100 pasados al programa, que indican la separación entre elementos del array que se suman.
        \item \texttt{OPT\_LEVEL}: El nivel de optimización del programa de C, mediante la flag -On, será un entero entre 0 y 3.
        \item \texttt{RANDOMIZE}: Un valor booleano que indica si se randomizará en el programa el acceso al vector de índices, lo cuál disminuye radicalmente la localidad de la operación.
    \end{itemize}
    
    \subsection{Relación con la localidad espacial, temporal y el prefetching}
    \label{cache}
    El prefetching \cite{wikiprefetch} es una técnica presente en prácticamente todos los CPUs actuales que funciona cargando varias líneas de la caché extras cuándo se solicita alguno de los datos por parte del CPU. De esta forma, si el programa respeta la localidad espacial y temporal, el prefetching asegura que muchos de estos datos ya estén presentes en la caché de antemano y no haya que volver a memoria principal para buscarlos, aumentando el rendimiento. \\
    
    Al aumentar los valores de D (la separación en doubles de cada elemento consultado), reduciremos la localidad espacial de los datos puesto que los accesos se realizarán a posiciones más alejadas de la memoria. Este empeoramiento de la localidad espacial afecta a los ciclos por consulta puesto que existen menos posibilidades de que al traer una línea o conjunto de memoria a la caché se encuentre el dato que se necesitará a continuación, en cuyo caso habrá que traerlo de memoria principal. \\
    
    Existe otro parámetro que afecta mucho a esta localidad y también al prefetching, \texttt{RANDOMIZE}. Cuándo este parámetro está activo, los accesos al array \texttt{A} pasan de ser lineales de la forma \texttt{A[0], A[D], A[2*D], ...}  a ser aleatorios como \texttt{A[8*D], A[2*D], A[120*D], ...} esto causa que cada elemento a sumar este en promedio mucho más separado de los siguientes, lo cuál invalida la técnica de prefetching, al no servir de nada cuando los datos están demasiado separados y se encuentran fuera del rango de precarga. \\
    
    \subsection{Realización de las mediciones}
    \label{metodologia}
    A partir de esos argumentos, y usando los valores de L dados en el guión de la práctica, realizamos $ 5 * 7 = 35 $ ejecuciones del programa (5 valores de D y 7 de L distintos). Además para cada de los valores anteriormente mencionados, repetimos las pruebas 10 veces. Tras guardar los resultados de estas pruebas en los archivos \texttt{resultsi.csv}, que son archivos \texttt{csv} dónde cada línea contiene los parámetros del programa D, R, L y los tiempos de ejecución obtenidos ck y ck\_medio para cada una de las 35 ejecuciones. \\
    
    Posteriormente calculamos la mediana del número de ciclos para cada par (D, L) y así evitamos que los datos dependan fuertemente de la ejecución concreta y la influencia de otros procesos. Después de realizar la mediana guardamos los datos en el archivo de resultados final \texttt{results.csv} dentro de la una carpeta cuyo nombre refleja los anteriores parámetros del script \texttt{D, OPT y RANDOMIZE}.\par
	
	\section{Características del ordenador}
	% Aqui meter las caracteristicas del CPU y la cache
	Las características generales del equipo en el que se realizaron las pruebas son las siguientes:
	
	\begin{itemize}
        \item OS: Ubuntu 20.04.4 LTS x86\_64 
        \item Kernel: 5.13.0-35-generic 
        \item Intel i5-8300H (4) @ 2.303GHz
        \item Memoria RAM: 3925MiB
    \end{itemize}
    
    Las características de la cache L1 (de datos, la de instrucciones es igual) de cada núcleo son las siguientes:
    
	\begin{itemize}
        \item Tamaño: 128 KiB (131072 bytes)
        \item Tamaño de línea: 64 bytes \textit{= CLS}
        \item Número de líneas: 512 \textit{= S1}
        \item Tipo: Asociativa por conjuntos, 8 lineas por conjunto, 64 conjuntos
    \end{itemize}
    
    y las de la cache L2 son las siguientes: 
    
	\begin{itemize}
        \item Tamaño: 1 MiB (1048576 bytes)
        \item Tamaño de línea: 64 bytes \textit{= CLS}
        \item Número de líneas: 4096 \textit{= S2}
        \item Tipo: Asociativa por conjuntos, 8 lineas por conjunto, 64 conjuntos
    \end{itemize}
    
    Como vemos, hemos obtenido los parametros CLS, S1 y S2 que fueron usados en el programa a partir de esto. La informacion de la CPU fue obtenida mediante el comando \texttt{neofetch}, la informacion principal de la cache fue obtenida en la funcion \texttt{info\_cache} en el script bash adjunto y resumida en el archivo generado por este \texttt{cache\_info.txt}, por ultimo, el tipo de la cache fue obtenido mediante el comando \texttt{hardinfo}.
    
	\section{Resultados}
	
	Las gráficas de esta sección se pueden observar con más detalle al final del documento en la sección de Gráficas Obtenidas \autoref{graficas}.Para obtenerlas, usamos un script de \texttt{RStudio} a partir de los datos obtenidos y recogidos en el archivo \texttt{results.csv} (ver\autoref{metodologia}). 

	\subsection{Test 1: Comparaciones de D y L con ck\_medios}
	En este primer test los parámetros son \texttt{D = \{1, 4, 8, 16, 80\}}, \texttt{OPT\_LEVEL = 0} y \texttt{RANDOMIZE = 0}, es decir, tenemos un test básico sin randomizar el array de índices ni aplicar optimizaciones de \texttt{gcc}. \\
	
	En la \textbf{Figura1} relacionamos los valores de L con los \texttt{Ck\_medios} obtenidos. Podemos observar que al consultar más líneas de caché el número de ciclos por acceso aumenta, esto se debe a que al pasar de cierto valor de L, llegamos a ocupar toda la memoria caché y se dan fallos de capacidad. Los valores que tomamos para L son \texttt{\{0.5*S1, 1.5*S1, 0.5*S2, 0.75*S2, 2*S2, 4*S2, 8*S2\}} siendo S1 el número de líneas que caben en la caché de nivel 1 y S2 las que caben en la de nivel 2. Entonces podemos ver que los saltos más grandes se dan cuando superamos la capacidad de una memoria caché. En la \textbf{Figura2} podemos distinguir cada línea según el valor de L. Las líneas amarilla y la naranja se corresponden a un L que no supera S1, siendo su tiempo de ejecución muy pequeño. Las verdes están entre S1 y S2, y tienen un tiempo de ejecución intermedio que sobre todo aumenta a partir de D = 16. Por último, las azules y moradas tienen valores que superan la capacidad de las dos cachés y podemos observar que son las ven más un incremento realmente exponencial de ciclos medios a partir de D = 8. En cualquier caso, al aumentar el valor de D (la separación en doubles entre cada dato tomado), el número de ciclos también es mayor, esto sucede dado que al acceder a posiciones de memoria muy separada, la localidad se reduce mucho, por lo que hay que realizar más accesos a la memoria principal. \\
	
	En la \textbf{Figura 2} relacionamos el valor de D con los ciclos. Como explicamos para la anterior gráfica, el acceso a los datos empeora al aumentar la separación puesto que la localidad se reduce y entonces obtenemos peores resultados.\\
	
	\begin{figure}[h!] 
        \centering
        \includegraphics[width=\linewidth]{grafica_agrup_D_1.png}
        \caption{Relación de los ciclos de ejecución con D}
    \end{figure}
	
	\begin{figure}[h!] 
        \centering
        \includegraphics[width=\linewidth]{grafica_agrup_L_1.png}
        \caption{Relación de los ciclos de ejecución con L}
    \end{figure}
    
	En la última figura de este apartado podemos observar desde un punto de vista paramétrico las magnitudes D (eje X), L (eje Z) y ck\_medios (eje Y). De esta forma podemos ver en una misma gráfica los 3 valores que nos interesan en cada uno de los 35 tests. \\
    
    \begin{figure}[h!] 
        \centering
        \includegraphics[width=\linewidth]{grafica_3d.png}
        \caption{Relación de los ciclos de ejecución con L y D}
    \end{figure}
    
	\newpage
	% Aqui meter las graficas comentadas y lo que sacamos de ellas
	Tras las ejecuciones explicadas anteriormente, los resultados obtenidos en \texttt{results.csv} se procesan en un script de \textsf{R} llamado \texttt{graficas.R}. En estas gráficas se intentaron relacionar los valores del número de lineas de cache consultadas \texttt{L}, el espacio entre elementos \texttt{D} y el número de ciclos medios en la ejecución del programa \texttt{ck\_medio}.
	
	
	
    \subsection{Test 2: Pruebas con optimización y randomización}
    
    En este apartado, se han realizado distintas pruebas utilizando accesos aleatorios al vector de índices y optimización del gcc con la opción -on. Para obtener las gráficas que referenciaremos a continuación, se utilizaron valores de D distintos a los del anterior apartado. \\
    
    En primer lugar, se comparó el efecto de la randomización en el acceso al vector ind, del que obtenemos los índices de los elementos consultados. En la \textbf{Figura 3} no estamos usando la randomización, mientras que en la \textbf{Figura 4} la activamos. Se observa claramente en la escala de la gráfica que los tiempos de ejecución son mucho más pequeños (de media 10 ciclos menos por acceso a memoria) cuando el array no está randomizado. Esto refuerza la hipótesis sobre el efecto de la localidad espacial y el prefetching en la eficiencia de un programa como vimos en el apartado sobre estos mecanismos de la memoria caché \autoref{cache}.
    
	\begin{figure}[h!] 
        \centering
        \includegraphics[width=\linewidth]{O0_R0_grafica_agrup_L.png}
        \caption{Relación de los ciclos de ejecución con L, array de índices no randomizado}
    \end{figure}
	\begin{figure}[h!] 
        \centering
        \includegraphics[width=\linewidth]{O0_R1_grafica_agrup_L.png}
        \caption{Relación de los ciclos de ejecución con L, array de índices randomizado}
    \end{figure}
    
    \subsection{Test 3: Pruebas con optimización y randomización}
    En este último test, veremos que al utilizar la optimización del compilador \texttt{gcc}mediante el argumento \texttt{-O3} (en comparación con no optimizar, \texttt{-O0}), el número de ciclos medios disminuye drásticamente. En la \textbf{Figura 5} no optimizamos el programa, mientras que en la \textbf{Figura 6} si lo hacemos. Los resultados arrojan un speedup de alrededor del 100\%, es decir, la eficiencia del programa se duplica al usar \texttt{-O3}. \\
    
    Estas optimizaciones se verán en mayor detalle en la siguiente práctica pero nos pareció interesante incluirlas aquí también al tener un efecto tan grande sobre los resultados en este programa. \\
    
    \begin{figure}[h!] 
        \centering
        \includegraphics[width=\linewidth]{O0_R0_grafica_agrup_D.png}
        \caption{Relación de los ciclos de ejecución con D, sin optimizaciones}
    \end{figure}
    
    \begin{figure}[h!] 
    	\centering
            \includegraphics[width=\linewidth]{O3_R0_grafica_agrup_D.png}
            \caption{Relación de los ciclos de ejecución con D, con optimizaciones}
    \end{figure}

    % Conclusion final
    \vfill\null
    \section{Conclusiones}
    Tras las pruebas realizadas variando los parámetros L, D y con la optimización de \texttt{gcc} llegamos a la conclusión de que efectivamente los peores resultados se obtienen con valores que nos dejan una baja localidad de los datos puesto que así son necesarios más accesos a memoria principal. \\
    Además al randomizar el acceso a los datos vemos que la ejecución es más lenta, ya que empeoramos los efectos de la precarga debido a que resulta imposible predecir los datos que serán utilizados a continuación.\\
    
    % Por si hace falta meter cuadros
    \newpage
    \onecolumn
    \printbibliography
	\appendix 
	\section{Código del programa}
	
	\Large{\texttt{main.c}:}
	\begin{lstlisting}
    void start_counter();
        double get_counter();
        double mhz();
        
        static unsigned cyc_hi = 0;
        static unsigned cyc_lo = 0;
        
        #include <stdlib.h>
        #include <stdio.h>
        #include <time.h>
        #include <pmmintrin.h>
        #include <ctype.h>
        #include <wait.h>
        #include <unistd.h>
        #include <math.h>
        
        // Constantes del CPU
        int CLS, S1, S2;
        
        // Parametros del test
        int D, L, R;
        
        // Archivo de salida (csv)
        FILE *resultsFile;
        
        int main(int argc, char **argv)
        {
            if (argc < 7)
            {
                printf("Uso correcto ./main D L CLS S1 S2 resultsFile");
                exit(EXIT_FAILURE);
            }
            else
            {
                // D: Salto entre elementos
                D = atoi(argv[1]);
                // L: Numero de lineas cache a usar
                L = atoi(argv[2]);
                // CLS: Tamaño de linea en bytes
                CLS = atoi(argv[3]);
                // S1: Numero de lineas en la L1
                S1 = atoi(argv[4]);
                // S2: Numero de lineas en la L2
                S2 = atoi(argv[5]);
                // Hallar R: numero de elem a sumar
                if (D <= 0 || L <= 0 || CLS <= 0 || S1 <= 0 || S2 <= 0) // Comprobamos que los valores de entrada son validos
                {
                    printf("Alguno de los argumentos numericos es incorrecto\n");
                    exit(EXIT_FAILURE);
                }
        
                /*
                 * Calculo del numero de elementos en un array
                 * Si D es mayor que el numero de doubles que caben en una linea cache, entonces cada elemento se consultara
                 * en una nueva linea, asi que el numero de lineas visitadas sera igual que el numero de elementos a los que accedemos
                 *
                 * En caso de que el valor de D sea menor, entonces realizamos un calculo para tener en cuenta que habra elementos
                 * que se consultan de una misma linea
                 */
                if (D > (CLS / sizeof(double)))
                    R = L;
                else
                    R = (int)ceil((double)(L * CLS) / (double)(D * sizeof(double)));
            }
        
            double *A, S[10], S_medio;
            int ind[R];
            double ck, ck_medio;
            int accesos = 10 * R;
        
            /*
             * Reserva de memoria para el vector A
             * El espacio reservado es igual al numero de elementos consultados por el espacio que necesita cada uno,
             * esto es, el tamaño de un double por D, el numero de doubles que ocupa cada uno.
             * Alineamos esta reserva de memoria con el CLS (tamaño de linea) para asegurarnos de que consultamos el
             * numero de lineas deseado.
             */
            A = (double *)_mm_malloc(R * D * sizeof(double), CLS);
        
            // Inicializamos el vector ind y los numeros aleatorios de A
            srand(time(NULL));
            for (int i = 0; i < R; i++)
            {
                ind[i] = i * D;
                A[ind[i]] = (((double)rand() / RAND_MAX) + 1.0); // double en rango [1,2)
                if (rand() / RAND_MAX < 0.5)
                {
                    A[ind[i]] = -A[ind[i]]; // signo aleatorio
                }
            }
        
            // Empezamos a contar el tiempo
            start_counter();
        
            // Repetimos la reduccion del vector 10 veces
            for (int i = 0; i < 10; i++)
            {
                S[i] = 0;
                for (int j = 0; j < R; j++)
                {
                    S[i] += A[ind[j]];
                }
            }
        
            // Calculo del valor medio de las S, para evitar que el compilador no optimize agresivamente y acabe sin calcular los resultados S[i]
            for (int i = 0; i < 10; i++)
            {
                S_medio += S[i];
            }
            S_medio /= 10;
        
            // Obtenemos los resultados temporales
            ck = get_counter();
            ck_medio = (double)(ck / accesos);
        
            // Imprimimos datos y resultados
        
            // Abrimos el archivo de resultados para meter la linea correspondiente a esta ejecucion
            resultsFile = fopen(argv[6], "a");
            if (resultsFile == NULL)
            {
                printf("El archivo de resultados \"%s\" no existe!\n", argv[6]);
                exit(EXIT_FAILURE);
            }
            // Formato de linea: D, R, L, ck, ck_medio
            fprintf(resultsFile, "%d,%d,%d,%d,%lf\n", D, R, L, (int)ck, ck_medio);
        
            // Tambien imprimimos por consola (o redirigimos con un pipe) informacion mas detallada acerca de esta
            printf("Resultados impresos al fichero de resultados con exito");
            printf("DEBUG INFO:\n");
            printf("Argumentos del test:\n");
            printf("\tNumero de elementos consultados: R = %d\n", R);
            printf("\tSalto entre elementos consultados: D = %d (%ld bytes)\n", D, (D * sizeof(double)));
            printf("\tNumero de lineas accedidas: L = %d\n", L);
            printf("\nPropiedades del ordenador:\n");
            printf("\tTamaño de linea: CLS = %d\n", CLS);
            printf("\tNumero de lineas en cache nivel 1: S1 = %d\n", S1);
            printf("\tNumero de lineas en cache nivel 2: S2 = %d\n", S2);
            printf("\nTiempos y ciclos:\n");
            printf("\tCiclos de ejecucion totales = %1.10lf\n", ck);
            printf("\tTiempo medio de acceso = %1.10lf ciclos\n", ck_medio);
        
            // Imprimimos el vector S y el S_medio
            printf("\nVector de resultados:\n");
            for (int i = 0; i < 10; i++)
            {
                printf("\tS[%d]: %lf\n", i, S[i]);
            }
            printf("S medio = %lf\n", S_medio);
        
            // Liberamos la memoria asignada y acabamos
            _mm_free(A);
            printf("\nMemoria liberada y programa terminado con exito!\n");
            return EXIT_SUCCESS;
        }
        /* Omitido código de medida de ciclos */
        \end{lstlisting}
        
        
        \Large{\texttt{mediana.c}}
        \begin{lstlisting}
        /**
         * @author Pablo Landrove (@PabloLandro)
         * @author Pablo Diaz (@derivada)
         * @brief Practica 1 Arquitectura de Computadores - Programa para calcular la mediana de los resultados
         * @date 2022-02-16
         *
         * NOTA: Compilar con las flags -O0 y -msse3
         */
        
        #include <stdlib.h>
        #include <stdio.h>
        #include <math.h>
        #include <string.h>
        
        void insertionSortInt(int array[], int size);
        void insertionSortDouble(double array[], int size);
        
        int main(int argc, char **argv)
        {
            FILE *datos, *result;
            size_t size = 100;
            char *line = malloc(size * sizeof(char));
            double ck_medios[35][10];
            int cks[35][10];
            int pts[35][3];
            char *filename = (char *)malloc(100 * sizeof(char));
        
            for (int i = 1; i <= 10; i++)
            {
                sprintf(filename, "results%d.csv", i);
                if ((datos = fopen(filename, "r")) == NULL)
                {
                    perror("Error al abrir el archivo");
                    exit(EXIT_FAILURE);
                }
                int j = 0;
                getline(&line, &size, datos);
                while (getline(&line, &size, datos))
                {
                    //printf("LINE: %s\n", line);
                    char *tok;
                    tok = strtok(line, ",");
                    pts[j][0] = atoi(tok);
                    tok = strtok(NULL, ",");
                    pts[j][1] = atoi(tok);
                    tok = strtok(NULL, ",");
                    pts[j][2] = atoi(tok);
                    tok = strtok(NULL, ",");
                    cks[j][i - 1] = atoi(tok);
                    tok = strtok(NULL, ",");
                    ck_medios[j][i - 1] = strtod(tok, NULL);
                    j++;
                    if (j >= 35)
                        break;
                }
                fclose(datos);
            }
        
            result = fopen("results.csv", "w");
            fprintf(result, "D,R,L,ck,ck_medio\n");
        
            for (int j = 0; j < 35; j++)
            {
                insertionSortInt(cks[j], 10);
                insertionSortDouble(ck_medios[j], 10);
                int ck_mediana = (cks[j][4] + cks[j][5]) / 2;
                double ck_medios_mediana = (ck_medios[j][4] + ck_medios[j][5]) / 2;
                fprintf(result, "%d,%d,%d,%d,%lf\n", pts[j][0], pts[j][1], pts[j][2], ck_mediana, ck_medios_mediana);
            }
        
            fclose(result);
            exit(EXIT_SUCCESS);
        }
        
        void insertionSortInt(int array[], int size)
        {
            for (int step = 1; step < size; step++)
            {
                int key = array[step];
                int j = step - 1;
        
                // Compare key with each element on the left of it until an element smaller than
                // it is found.
                // For descending order, change key<array[j] to key>array[j].
                while (key < array[j] && j >= 0)
                {
                    array[j + 1] = array[j];
                    --j;
                }
                array[j + 1] = key;
            }
        }
        
        void insertionSortDouble(double array[], int size)
        {
            for (int step = 1; step < size; step++)
            {
                double key = array[step];
                int j = step - 1;
        
                // Compare key with each element on the left of it until an element smaller than
                // it is found.
                // For descending order, change key<array[j] to key>array[j].
                while (key < array[j] && j >= 0)
                {
                    array[j + 1] = array[j];
                    --j;
                }
                array[j + 1] = key;
            }
        }
\end{lstlisting}
\newpage

\section{Gráficas obtenidas}

A continuación presentamos las gráficas anteriores en  formato de una columna para que sean más fáciles de visualizar. \\

\setcounter{figure}{0}   
\label{graficas}
	\begin{figure}[!ht]
        \centering
        \includegraphics[width=\linewidth]{grafica_agrup_D_1.png}
        \caption{Relación de los ciclos de ejecución con D}
            \label{Figura1}
    \end{figure}
	
	\begin{figure}[!ht]
        \centering
        \includegraphics[width=\linewidth]{grafica_agrup_L_1.png}
        \caption{Relación de los ciclos de ejecución con L}
        \label{Figura2}
    \end{figure}
    
    \begin{figure}[!ht]
        \centering
        \includegraphics[width=\linewidth]{grafica_3d.png}
        \caption{Relación de los ciclos de ejecución con L y D}
        \label{Figura3}
    \end{figure}
    
    \begin{figure}[!ht]
        \centering
        \includegraphics[width=\linewidth]{O0_R0_grafica_agrup_L.png}
        \caption{Relación de los ciclos de ejecución con L, array de índices no randomizado}
        \label{Figura4}
    \end{figure}
    
    \begin{figure}[!ht]
        \centering
        \includegraphics[width=\linewidth]{O0_R1_grafica_agrup_L.png}
        \caption{Relación de los ciclos de ejecución con L, array de índices randomizado}
        \label{Figura5}
    \end{figure}
    
    \begin{figure}[!ht]
        \centering
        \includegraphics[width=\linewidth]{O0_R0_grafica_agrup_D.png}
        \caption{Relación de los ciclos de ejecución con D, sin optimizaciones}
        \label{Figura6}
    \end{figure}
    
    \begin{figure}[!ht]
        \centering
        \includegraphics[width=\linewidth]{O3_R0_grafica_agrup_L.png}
        \caption{Relación de los ciclos de ejecución con D, con optimizaciones}
        \label{Figura7}
    \end{figure}
\end{document}